% Zusammenfassung und Resumee der Diplomarbeit:
\section{Zusammenfassung}

Hier steht die Zusammenfassung und das Resumee der Diplomarbeit!
\newpage

% Anhang zur Diplomarbeit:
\section{Anhang}

Hier stehen zusätzliche Informationen zur Diplomarbeit!
Zum Beispiel können hier Listings oder Teile des Programmes abgebildet sein:

\begin{lstlisting}
// create new document
string path = @"C:\Data\sample.xlsx";
// add title row 
List<string> titleRow = new List<string>();
titleRow.Add("This is the 1st cell");
titleRow.Add("This is the 2nd cell");
openXmlExcel.addTitleRow(titleRow);
// add scatter chart
List<double[]> xValues = new List<double[]>();
List<double[]> yValues = new List<double[]>();
List<string> lineNames = new List<string>();
// first line
xValues.Add(new double[5] { -2, -1, 0, 1, 2 });
yValues.Add(new double[5] { -2, -1, 0, 1, 2 });
lineNames.Add("Line 1");
// second line
xValues.Add(new double[5] { -2, -1, 0, 1, 2 });
yValues.Add(new double[5] { -3, -1, 0, 1, 3 });
lineNames.Add("Line 2");
// save document 
openXmlExcel.saveFile();
\end{lstlisting}

\newpage

\section{Verzeichnisse}
% Tabellenverzeichnis:
\listoffigures 
%
% Abkürzungsverzeichnis:

% Abkürzungsverzeichnis:
\section*{Abkürzungen}
% usage: \acs{DA}
\begin{acronym}[22]
\setlength{\itemsep}{-0.3 \parsep}
% Einträge für Abkürzungen:
\acro{DA}[DA]{Diplomarbeit}
\acro{ABK}[ABK]{Abkürzung}
% weitere Einträge ...
\end{acronym}

%
% Literaturverzeichnis:
%
%-----------------------------------------------
% Bibliographie, wenn notwendig. 
% Ansonsten: Auskommentieren!
%
\bibliographystyle{unsrt} 		%Sortiert die Reihenfolge der Literaturstellen nach Verwendungsreihenfolge
\bibliography{DA_05_LitVerzeichnis}
%\input{./../02_DOC_Main/DA_05_LIT.tex}