%
%****************************************************************************************************
%================================ B E G I N  1.   P A G E =============================
%****************************************************************************************************
%
\thispagestyle{ErsteSeite}
$~$
\vspace{2.0cm}

\begin{center}
\textcolor{dred} {\resizebox{10cm}{1.0cm}{\textbf{DIPLOMARBEIT}}}
\vspace{1.5cm}

\begin{tabular}{p{15.0cm}} 
\begin{center}\textcolor{black!}{\LARGE\textbf{\DAName}}   \\ \end{center}
\end{tabular}
\end{center}

\vspace{1.0cm}
\begin{center}
\includegraphics[scale=1.5]{../../03_AUX/01_PICT/LogoDerDA.pdf}
\end{center}
\vfill

\changefont{pag}{m}{n}

\begin{tabular}{p{6cm} p{4cm} p{5cm}}
\textbf{Ausgeführt von:} & \textbf{Jahrgang/Klasse:} & \textbf{Betreuer:} \\ 
\DPLNameOne & \SYear/\SClass & \BNameOne \\ 
\DPLNameTwo&  \SYear/\SClass & \BNameTwo \\ 
&&\\
\textbf{Projektpartner:} &&\\
Firmenname 1 &Ansprechpartner 1& Funktion 1\\
& Ansprechpartner 2& Funktion 2\\
&&\\
&&\\ \hline
\textbf{Abgabevermerk:} & &\\ 
&&\\
Datum: & Unterschrift:& \\ 
&&\\
\end{tabular}

\newpage
%
%****************************************************************************************************
%============================= B E G I N   2.    P A G E ==============================
%****************************************************************************************************
%
\thispagestyle{ErsteSeite}
\addsec*{Eidesstattliche Erklärung}

\begin{quote}
Ich erkläre an Eides statt, dass ich die vorliegende Diplomarbeit selbst\-ständig und ohne fremde Hilfe verfasst, andere als die angegebenen Quellen und
Hilfsmittel nicht benutzt und die den benutzten Quellen wörtlich und inhaltlich entnommenen Stellen als solche erkenntlich gemacht habe.
\\[4\baselineskip]
\end{quote}

\begin{center}
\begin{tabular}{p{10cm} p{4cm}}
Weiz, am \ADatum$~$\DPLNameOne:  & \dotfill \\ 
 &  \\ 
 & \\
Weiz, am \ADatum$~$\DPLNameTwo:& \dotfill \\ 
\end{tabular} 
\end{center}
\newpage

%
%****************************************************************************************************
%=============================== B E G I N   3.   P A G E ============================
%****************************************************************************************************
%
\thispagestyle{ErsteSeite}
\addsec*{Kurzbeschreibung}
Die Kurzbeschreibung der Arbeit  ist eine sehr prägnante Inhaltsangabe, mit wichtigen Eigenschaften und Beschreibungen der in der Diplomarbeit behandelten Themengebieten. Der Umfang sollte \textbf{eine Seite} nicht überschreiten!
\vspace{3.0cm}

\textbf{Das entwickelte Gerät soll folgende Funktionen aufweisen:}
\begin{itemize}
	\item Eigenschaft 1
	\item Schutzklasse mindestens IP44
\end{itemize}
\newpage

%
%****************************************************************************************************
%=============================== B E G I N   4.   P A G E ============================
%****************************************************************************************************
%
\thispagestyle{ErsteSeite}
\addsec*{Abstract}
Das Abstract ist die Kurzbeschreibung der Arbeit  in Englisch verfasst. Ein Abstract ist eine Inhaltsangabe, die sehr prägnant verfasst ist. Der Umfang sollte \textbf{eine Seite} nicht überschreiten!
\vspace{3.0cm}

\textbf{The developed device will feature the following items:}
\begin{itemize}
	\item Specification 1 
	\item Minimal protection class of IP44
\end{itemize}
\newpage

%
%****************************************************************************************************
%=============================== B E G I N   5.   P A G E ============================
%****************************************************************************************************
%
\thispagestyle{ErsteSeite}
\addsec*{Vorwort}
Im Vorwort soll eine kurze Beschreibung des schulischen Umfeldes stehen; per\-sön\-liche Vorstellungen können ebenfalls enthalten sein. Im Vorwort können auch Gründe für die Wahl des Themas, Angaben zu einem persönlichen Bezug und ähnliches aufgeführt werden. Das Vorwort ist auch der Platz für Danksagungen.
Das Vorwort endet mit dem Datum und dem Namen des Autors bzw. der Autorin.\\
Um die Diplomarbeit möglichst reibungsfrei und effektiv bearbeiten zu können, sollten Sie die nachfolgenden Punkte schon zu Beginn beachten.
\vspace{2.0cm}

\begin{center}
\begin{tabular}{p{10cm} p{6.5cm}}
Ort, am \ADatum$~$\DPLNameOne\\ 
 &  \\ 
Ort, am \ADatum$~$\DPLNameTwo\\ 
\end{tabular} 
\end{center}
\newpage
\newpage
%
%
%****************************************************************************************************
%=============================== B E G I N   6.   P A G E ============================
%****************************************************************************************************
%
%
\thispagestyle{ErsteSeite}
\tableofcontents
%