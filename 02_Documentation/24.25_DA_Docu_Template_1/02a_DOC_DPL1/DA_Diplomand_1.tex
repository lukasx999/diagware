%
%****************************************************************************************************
%======================= B E G I N   D O C U M E N T   S E T T I N G S ========================
%****************************************************************************************************
%
\documentclass[a4paper, 12pt, xcolor=dvipsnames]{scrartcl}		% Article als Modul
\input{./../00_LATdef/UsePackages.tex}                    			% Input file für gemeinsame Package Definitionen
%
%--------------------Set data for document:
\newcommand{\SYear}{2024/25}
\newcommand{\SClass}{5xHET}
\newcommand{\ADatum}{04. April 2025}
\newcommand{\DAName}{Mustervorlage zur Erstellung einer Diplomarbeitsdokumentation mit \LaTeX{}}
\newcommand{\DPLNameOne}{Vorname Nachname1}		% Diplomand 1
\newcommand{\DPLNameTwo}{Vorname Nachname2}		% Diplomand 2
\newcommand{\BNameOne}{Vorname Nachname1}			% Betreuer 1
\newcommand{\BNameTwo}{Vorname Nachname2}			% Betreuer 2
%
%\setboolean{UseWm}{true} 								% Wasserzeichen einschalten
%\newcommand{\WatermarkName}{Korrekturexemplar}			%Text für das Wasserzeichen
\newcommand{\WatermarkName}{Diplomand 1}				%Text für das Wasserzeichen
%
%==============================S E T T I N G S ===================================
%
\definecolor{green}{rgb}{0.8,1.0,0.2}
\definecolor{orange}{rgb}{0.9,0.64,0.25}
\definecolor{kaki}{rgb}{0.74,0.74,0.088}
\definecolor{blue}{rgb}{0.2,0.59,0.94}
\definecolor{dred}{rgb}{0.722,0.18,0.18}
%
%------------------------------Watermark wird mit einer Designvariablen geladen ---------------------------------------------------------------------------------
%
\ifthenelse{\boolean{UseWm}}%
{\usepackage{draftwatermark}\SetWatermarkText{\WatermarkName$~-~$\WatermarkName\xspace$~-~$\WatermarkName} \SetWatermarkLightness{0.8}\SetWatermarkScale{2}}{}
%
\graphicspath{{./../03_AUX/01_PICT/}}
\newcommand{\changefont}[3]{\fontfamily{#1} \fontseries{#2} \fontshape{#3} \selectfont} % Change a new font
%
% --------------------Set the figure description:
%
\captionsetup{margin=10pt,font=small, labelfont={color=black!80, bf}, textfont={color=black!60}, format=hang, indention=-1cm}
\captionsetup[wrapfigure]{name=Bild}
\captionsetup[figure]{name=Bild}
% --------------------Set the page margins:
%
\setlength{\voffset}{1.0cm}
\setlength{\textwidth}{16.0cm}
%\addtolength{\textheight}{10.0cm}
\setlength{\textheight}{23.2cm}
\setlength{\topmargin }{-1.5cm}
\setlength{\marginparsep }{0pt}
\setlength{\headsep }{1.2cm}
\setlength{\oddsidemargin}{0.4cm}
\setlength{\evensidemargin}{0.4cm}
\setlength{\marginparwidth}{0pt}
\setlength{\hoffset}{-12pt}
\setlength{\headwidth}{16.0cm}
\setlength{\footskip}{40pt}
\setlength{\parindent}{0pt}
%
%

%--------------------Set environment for Header and and Footer:
%
\pagestyle{fancy}
\rhead{\includegraphics[scale=0.07]{/HTL_logo.pdf}}
\chead{}
\lhead{\begin{tabular}{p{7.0cm}} 
\small{\changefont{pag}{b}{sc}\textcolor{black!30}{\DAName}}
\end{tabular} \vspace{0.1cm}}
\lfoot{\small{\changefont{pag}{b}{sc}\textcolor{black!30}{\SYear - \SClass - \DPLNameOne, \DPLNameTwo}}} 
\cfoot{}
\rfoot{\small{\changefont{pag}{b}{sc}\textcolor{black!30}{\thepage\ \textbf{\textcolor{orange!100}I} \pageref{LastPage}}}}
\renewcommand{\headrulewidth}{0.4pt}
\renewcommand{\footrulewidth}{0.4pt}
\let\myHeadrule\headrule
\let\myFootrule\footrule
\renewcommand\headrule{\color{black!30}\myHeadrule }
\renewcommand\footrule{\color{black!30}\myFootrule }
\addto{\captionsngerman}{%
  \renewcommand*{\contentsname}{Inhalt}
  \renewcommand*{\listfigurename}{Abbildungen}
  \renewcommand*{\listtablename}{Abbildungsverzeichnis}
}
%\pagestyle{fancy}
\fancypagestyle{ErsteSeite}{%
   \fancyhf{}%
   \fancyhead[L]{}
}

%\makenoidxglossaries
\renewcommand*\acronymname{Vokabelverzeichnis}
\renewcommand*\glossaryname{Vokabelverzeichnis}

%--------------------Neue Umgebung für M E R K S A T Z:
%
%-------------------- Zählerdefinition:
%
\newcounter{MsNr}
\renewcommand{\theMsNr}{\arabic{MsNr}}
%
\newlength{\boxw}
\newlength{\boxh}
\newlength{\shadowsize}
\newlength{\boxroundness}
\newlength{\tmp}
\newsavebox{\shadowblockbox}
\setlength{\shadowsize}{6pt}
\setlength{\boxroundness}{3pt}
%
\newenvironment{shadowblock}[1]%
{\begin{lrbox}{\shadowblockbox}\begin{minipage}{#1}}%
{\end{minipage}\end{lrbox}%
\settowidth{\boxw}{\usebox{\shadowblockbox}}%
\settodepth{\tmp}{\usebox{\shadowblockbox}}%
\settoheight{\boxh}{\usebox{\shadowblockbox}}%
\addtolength{\boxh}{\tmp}%
\begin{tikzpicture}
\addtolength{\boxw}{\boxroundness * 2}
\addtolength{\boxh}{\boxroundness * 2}
\foreach \x in {0,.05,...,1}
{
\setlength{\tmp}{\shadowsize * \real{\x}}
\fill[xshift=\shadowsize - 1pt,yshift=-\shadowsize + 
1pt,black,opacity=.04,rounded corners=\boxroundness] 
(\tmp, \tmp) rectangle +(\boxw+20 - \tmp - \tmp, \boxh+20 - \tmp - 
\tmp);
}
\filldraw[fill=yellow!10, draw=kaki!100, rounded corners=\boxroundness] (0, 
0) rectangle (\boxw+20, \boxh+20);
\draw node[xshift=\boxroundness,yshift=\boxroundness,inner sep=10pt,outer 
sep=0pt,anchor=south west] (0,0) {\usebox{\shadowblockbox}};
\end{tikzpicture}}
%
%====================================================================
\newenvironment{Merksatz}
{\begin{shadowblock}{15.0cm}
{\par\vspace*{0.0cm}
\noindent \refstepcounter{MsNr}
\textcolor{kaki!100} {\large{}\textbf{Merke \theMsNr :}} % Hier kann man statt Merke auch etwas anderes hinschreiben
\label{Msl-a}}
}
{\end{shadowblock}}
%
%--------------------End    M E R K S A T Z
% Settings for Codelistings within documentation:
\lstset{ % General setup for the package
	language=C,
	basicstyle=\small\sffamily,
	numbers=left,
 	numberstyle=\tiny,
	frame=tb,
	tabsize=4,
	columns=fixed,
	showstringspaces=false,
	showtabs=false,
	keepspaces,
	commentstyle=\color{red},
	keywordstyle=\color{blue}
}

\setcounter{footnote}{0}

\changefont{pag}{m}{n}                    % Input file für gemeinsame Definitionen
%
%****************************************************************************************************
%=========================== B E G I N   S U B D O C U M E N T  ==========================
%****************************************************************************************************
%
\begin{document}
\changefont{pag}{m}{n}
\section{Projektmanagement}
Um den zeitlichen Ablauf der Diplomarbeit zu planen und zu dokumentieren sind einigen Dokumente und Programme erforderlich.
\begin{enumerate}
\item\textbf{WORD File:} Das File \textbf{SJ.SJ\_DaName\_Antrag.docx} ist von Ihnen und von mir gemeinsam zu erstellen. Sie geben dann die Daten in das System ein.
\item\textbf{EXCEL File:} Das File \textbf{DA.2017.18.LogBook\_DA-Name\_Klasse\_0.xls} wird von Ihnen ge\-führt, und ist eine Art Tagebuch über den Verlauf der DA mit den eingetragenen Meilensteinen.
\end{enumerate}
\vspace{-0.6cm}



\section{Dokumentation der Diplomarbeit}
Erstellen Sie die Dokumentation entsprechend der Vorlagen der HTL-Weiz.\\
\textbf{Wieder gilt:} Ein früher Beginn sichert den Erfolg! Folgende Punkte sind wichtig:
\begin{enumerate}
\item\textbf{Dokumentstruktur:} Doppelseitig
\item\textbf{Schriftart:} Entsprechend dem Template in Latex
\item\textbf{Zeilenabstand:} 1.15
\end{enumerate}

\vspace{-0.6cm}

\subsection{Zeichnungen}
Ich empfehle die Zeichnungen, die für die Dokumentation der DA notwendig sind, selbst und mit der Software \textbf{VISIO} zu zeichnen. Die meisten bei mir verfügbaren Vorlagen sind in diesem Format.\\
Beginnen Sie schon früh mit dem Erstellen solcher Blockschaltbilder und technischer Zeichnungen, dann geht das Dokumentieren am Ende der DA umso schneller.\\
Vorlagen für diese Zeichnungen in VISIO können Sie bei mir beziehen.
\vspace{-0.6cm}

\subsection{Zitate und Quellenangaben}
Ein wesentliches Prinzip wissenschaftlichen Arbeitens ist die Nachvollziehbarkeit der in einer Diplomarbeit  getätigten Aussagen. Werden in einer schriftlichen Arbeit fremde Quellen verwendet, das heißt
zitiert beziehungsweise den eigenen Aussagen zugrunde gelegt, so sind diese Quellen vollständig und korrekt anzugeben.\\
Sie sollten in Ihre DA Dokumentation alle Stellen die Sie \textbf{direkt und wortgleich} aus einer fremden Quelle beziehen kennzeichnen!\\
Inhaltlich nacherzählte Passagen sind dabei \textbf{nicht anzugeben}. Finden Sie einen guten Mix zwischen Zitaten und erarbeitetem Wissen. Streben Sie bitte nach Quellen, die auch außerhalb des Internets zu finden sind (Bücher)!
Bei Bildunterschriften soll die Quellenangabe direkt mit Fußnote auf der selben Seite erfolgen.\\

Sie sollten folgende Dinge im Zusammenhang mit Zitaten unbedingt vermeiden:
\begin{enumerate}
\item zu viele und zu umfangreiche Zitate
\item unnötige Zitate (z. B. technisches Allgemeinwissen)
\item ungenaue und falsche Zitate
\item \textbf{zu wenige} Zitate (haben Sie Alles wirklich selbst entwickelt?). Ein Zitat kann dem Leser auch dazu dienen, weitere Quellen für  Recherchen zu erfahren.
\item aus ihrem Zusammenhang gerissene Zitate
\end{enumerate}
\vspace{-0.6cm}

\subsection{FactSheets}
Ein FS ist eine Broschüre zur Kurzinformation über ein Stoffgebiet und besitzt genau \textbf{2 Seiten}. Es fasst für den Schüler die wichtigsten Dinge für eine Laborübung kuz und prägnant zusammen.\\
FactSheets sind mit dem professionellen Textverarbeitungsprogramm \LaTeX zu erstellen. Eine geeignete Vorlage für die Dokumente bekommen Sie von mir.
\vspace{-0.6cm}

\subsection{Laborübungen}
Die Laborübungen sind vollständig mit Zielsetzung, Aufgabenstellung und Ausarbeitung zu erstellen. Es gibt von der Laborübung eine Lehrer und eine Schülerversion.\\
Laborübungen sind mit dem professionellen Textverarbeitungsprogramm \LaTeX zu erstellen. Eine geeignete Vorlage für die Dokumente bekommen Sie von mir.
\vspace{-0.6cm}

\subsection{Designreports}
Designreports sind mit dem professionellen Textverarbeitungsprogramm \LaTeX zu erstellen. Eine geeignete Vorlage für die Dokumente bekommen Sie von mir.
\vspace{-0.6cm}

\pagebreak

\section{Konstruktion und Entwurf von Baugruppen}
\subsection{Berechnungen}
Führen Sie die für die DA notwendigen Berechnungen mit dem Programm \textbf{MathCad} oder \textbf{SCILAB} durch. Beginnen Sie hier ebenfalls so früh wie möglich, um darin Routine zu bekommen. Eine Beschreibung für die Einführung in MathCad und SCILAB ist vorhanden.
\vspace{-0.6cm}

\subsection{Simulation elektronischer Schaltungen}
Ein kostenloses und sehr flexibles Programm Simulationen elektrischer Schaltungen durchführen zu können ist \textbf{LT-Spice} von Linear Technology. Eine Beschreibung für die Einführung in die Simulationstechnik ist im Internet vorhanden.
\vspace{-0.6cm}

\subsection{Platinenlayout}
Für die professionelle und automatische Erstellung von Layouts wird das Programm \textbf{EAGLE} in der Version 5.11 verwendet. Die SW kann mit der Schullizenz auch auf dem privaten Rechner installiert werden. Eine Dokumentation ist ebenfalls vorhanden.\\
Beachten Sie beim Entwurf des Layouts einer Schaltung Dokumente und Richtlinien, die Sie von mir bekommen. Darin werden Hinweise zum Entwurf, sowie das \textbf{EMV} und \textbf{ESD}-gerechte Leiterplattendesign behandelt.

%\vspace{-0.6cm}


\section{Kommunikation}
\subsection{Schriftliche Korrespondenz}
Die schriftliche Kommunikation wird hauptsächlich über email geführt. Beachten Sie folgende Punkte:
\begin{enumerate}
\item\textbf{Zieladresse:} Bitte senden Sie Ihre mails immer an: \textbf{Ewald.Bergler@htlweiz.at} - oftmals schreibe ich von anderen Adressen zurück. Verwenden Sie jedoch immer die HTL Addresse! Nur so ist es gewährleistet, dass ich die Nachricht so schnell als möglich bekomme.
\item\textbf{Addressaten:} Schreiben Sie Ihre Nachricht immer an \textbf{ALLE} die an Ihrer DA beteiligt sind, damit stellen Sie sicher, dass jeder den gleichen Informationsstand hat. Meine Telefonnummer lautet: \textbf{0664 963 4 752}.
\end{enumerate}
\textbf{Grundsätzlich:} Es werden keine Daten, Dokumente oder Files über email versendet $\to$ dazu gibt es Laufwerke in einer Cloud!\\
Zusätzlich muss ein Tagebuch über OneNote geführt werden, in dem ebenfalls eine Übergabe von Informationen stattfinden kann. Einen entsprechenden Link zum Tagebuch erstelle ich. 

\subsection{Datenaustausch}
Elektronische Daten werden \textbf{ausschließlich} über eine Ordnerstruktur in einer Cloud ausgetauscht. Dies dient der Sicherung der Datenkonsistenz und Übersichtlichkeit. 



\section{Beurteilung der Diplomarbeit}
\subsection{Schriftliche und praktische Beurteilungskriterien der Arbeit}
\begin{enumerate}
\item\textbf{Fachkompetenz:} Wie gut kennt sich der Diplomand im Fachgebiet aus!
\item\textbf{Methodenkompetenz:} Welche Methoden zur Problemlösung wurden gewählt!
\item\textbf{Selbstkompetenz:} Hier fließt Ihre Motivation und Termintreue ein!
\item\textbf{Sprachkompetenz:} Wie gut kann der Diplomand technische Zusammenhänge sprachlich richtig darstellen!
\item\textbf{Dokumentation:} Wie gut ist die Dokumentation der Arbeit ausgeführt!
\end{enumerate}

\subsection{Mündliche Beurteilungskriterien der Präsentation und Verteidigung}
\begin{enumerate}
\item\textbf{Zeitliche Einteilung:} Sie können Ihre Diplomarbeit im vorgegebenen Zeitrahmen darstellen.
\item\textbf{Inhalt:} Die inhaltlich-fachliche Auseinandersetzung mit dem Thema ist in der erforderlichen Tiefe gegeben.
\item\textbf{Argumentation:} Sie können Ihre Standpunkte überzeugend argumentieren.
\item\textbf{Rethorik:} Der Kandidat kann seine Diplomarbeit in Standardsprache klar strukturiert und verständlich darstellen.
\item\textbf{Medienumgang:} Sie können mit zeitgemäßen Präsentationsmitteln umgehen.
\end{enumerate}

\pagebreak

%
%Verzeichnisse (to be commented out for compilation of main file):
 %\input{./../02_DOC_Main/DA_03_ADDendum.tex}
%
%
%****************************************************************************************************
%============================= E N D   S U B D O C U M E N T  ==========================
%****************************************************************************************************
%
\end{document}


